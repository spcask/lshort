\documentclass[notitlepage]{report}
\author{Susam Pal}
\title{About \TeX}
\begin{document}
\maketitle

\begin{center}
This is an example document
\end{center}

\tableofcontents

\chapter{\TeX}
\section{Introduction}
\TeX{} is a typesetting system designed and mostly written by Donald
Knuth and released in 1978.

\section{Metafont}
Metafont, not strictly part of \TeX{}, is a font description system
which allows the designer to describe characters algorithmically. This
term derives from the fact that Metafont describes characters as having
been drawn by abstract brushes (and erasers).

\section{Macros}
\TeX{} provides an unusual macro language; the definition of a macro not
only includes a list of commands but also the syntax of the call. Macros
are completely integrated with a full-scale interpreted compile-time
language that also guides processing.

\section{Development}
The original source code for the current \TeX{} software is written in
WEB, a mixture of documentation written in \TeX{} and a Pascal subset in
order to ensure portability. For example, \TeX{} does all of its dynamic
allocation itself from fixed-size arrays and uses only fixed-point
arithmetic for its internal calculations. As a result, \TeX{} has been
ported to almost all operating systems, usually by using the web2c
program to convert the source code into C instead of directly compiling
the Pascal code.

\section{License}
Donald Knuth has indicated several times that the source code of TeX has
been placed into the ``public domain'', and he strongly encourages
modifications or experimentations with this source code. Since the code
is still copyrighted, it is technically free/open-source software and
not in the public domain in the legal sense.

\section{LaTeX}
\LaTeX{}, a shortening of Lamport \TeX{} is a document preparation
system. When writing, the writer uses plain text as opposed to
formatted text as users of word processors like Microsoft Word. The
writer uses markup tagging conventions to define the general structure
of a document (such as article, book, and letter), to stylise text
throughout a document (such as bold and italic), and to add citations
and cross-references. A \TeX{} distribution such as \TeX{} Live or
MikTeX is used to produce an output file (such as PDF or DVI) suitable
for printing or digital distribution.

\section{Compatibility}
\LaTeX{} documents (*.tex) can be opened with any text editor. They
consist of plain text and do not contain hidden formatting codes or
binary instructions. Additionally, \TeX{} documents can be shared by
rendering the \LaTeX{} file to Rich Text Format (.rtf) or XML. This can
be done using the free software programs LaTeX2RTF or TeX4ht. \LaTeX{}
can also be rendered to PDF files using the \LaTeX{} extension pdfLaTeX.
\LaTeX{} files containing Unicode text can be processed into PDFs by the
\LaTeX{} extension XeLaTeX.

\section{Related}
As a macro package, \LaTeX{} provides a set of macros for \TeX{} to
interpret. There are many other macro packages for \TeX{}, including
Plain \TeX{}, GNU Texinfo, AMSTeX, and ConTeXt. The default font for
\LaTeX{} is Knuth's Computer Modern, which gives default documents
created with \LaTeX{} the same distinctive look as those created with
plain \TeX{}. XeTeX allows the use of OpenType and TrueType (that is,
outlined) fonts for output files. There are also many editors for
\LaTeX{}.
\end{document}
